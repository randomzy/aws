
\documentclass[12pt]{article}

\usepackage[utf8]{inputenc}
\usepackage[bulgarian]{babel}
\usepackage{graphicx}
\usepackage{sidecap}  %required for side captions
\usepackage{amssymb}
\usepackage{amsmath}
\usepackage{hyperref}
\usepackage{commath}  
\usepackage[top=1.3in, bottom=1.5in, left=1.3in, right=1.3in]{geometry}


\begin{document}
	\begin{center}
        \LARGE{\textbf{Тема: @title-of-project}}
        
        \bigskip
        \Large{Предмет: @course-name}
        
        \medskip
        \Large{Изготвил: Мартин Танчев, фн: 81401, имейл: mtanchev@uni-sofia.bg}
        
        \medskip
        \Large{Лектор: Милен Петров, година: 2020}
        
        \bigskip
	\end{center}
    
    
  %  \newpage
    \tableofcontents
    \bigskip
    \bigskip
    \newpage
  
\section{Условие} 

\noindent @lorem25.

\medskip


\noindent @lorem25  \textbf{@Term1} @lorem3 \textbf{@Term2}.

\section{Въведение}

@lorem25

\noindent @lorem8:

\begin{itemize}
    \item @lorem2.
    
    \item @lorem7.
    
    \item @lorem6.
\end{itemize}

\section{Теория}
\noindent lorem10. 

\medskip

\noindent @Lorem6 \textb{Term3} @lorem15.

\noindent @Lorem6 \textbf{Term4}.

%\newpage

\section{Използвани технологии}
\begin{itemize}
    \item \textbf{@TermTech1} - version @term-version1.
    
    \item \textbf{@TermTech2} - @term-version2.
\end{itemize}

%\newpage

\section{Инсталация и настройки}
\noindent\textbf{Стъпка 1.} @lorem8.

\medskip

\noindent\textbf{Стъпка 2.} @lorem2 \textbf{@term}. @lorem4 \textbf{config}.

\begin{figure}[h!]
\centering
  \includegraphics[width=0.8\textwidth]{installation1}
  \caption{@FigCaption - [@CiteSource]}
\end{figure}


\noindent @lorem15.

\medskip

\section{Кратко ръководство за потребителя}

@lorem25.

\begin{figure}[h!]
\centering
    \includegraphics[scale=0.4]{9999_fig1.png}
  \caption{@FigCaption - [@CiteSource]}
\end{figure}

\noindent   @Lorem24:

\begin{SCfigure}
\centering
   \caption{aa@FigCaption - [@CiteSource]}
   \includegraphics[scale=0.5\textwidth]{9999_fig1.png}
\end{SCfigure}

\begin{SCfigure}
  \centering
  \includegraphics[width=0.7\textwidth]{9999_fig1.png}
    \caption{This is the same picture of the universe as above, but now the captions appear in the side next to the image}
\end{SCfigure}

\section{Примерни данни}

Lorem8: \textbf{@SampleData1}, \textbf{@SampleData12}, . 

 \noindent @lorem10:

% https://www.overleaf.com/learn/latex/Tables#List_of_tables


\begin{table}
\caption{@TableCaption}
\center
\begin{tabular}{ |c|c|c| }
 \hline
 cell1 & cell2 & cell3 \\ \hline
 cell4 & cell5 & cell6 \\ \hline 
 cell7 & cell8 & cell9 \\ 
 \hline
\end{tabular}
%\label{table:ta}
\end{table}
%\caption{\label{tab:table-name}Your caption.}

 
\section{Описание на програмния код}

\subsection{@Module1}
 @Lorem10.

\subsection{@Module2}

\medskip


\section{Приноси на студента, ограничения и възможности за бъдещо развитие}

@lorem25.

\medskip

\noindent @lorem26.

\section{Какво научих}
@lorem33.

\section{Списък с фигури и таблици}

\listoftables

\listoffigures

\section{Използвани източници}

\noindent\href{http://php.net/manual/en/function.password-hash.php}{[1] password hash}
 
\medskip



\bigskip

\end{document}